\documentclass[11pt]{article}
\usepackage[top=2.1cm,bottom=2cm,left=2cm,right= 2cm]{geometry}
%\geometry{landscape}                % Activate for for rotated page geometry
\usepackage[parfill]{parskip}    % Activate to begin paragraphs with an empty line rather than an indent
\usepackage{graphicx}
\usepackage{amssymb}
\usepackage{epstopdf}
\usepackage{amsmath}
\usepackage{multirow}
\usepackage{hyperref}
\usepackage{changepage}
\usepackage{lscape}
\usepackage{ulem}
\usepackage{multicol}
\usepackage{dashrule}
\usepackage[usenames,dvipsnames]{color}
\usepackage{enumerate}
\usepackage{amsmath}


\begin{document}


\begin{table}
\centering
%\footnotesize
\begin{tabular}{p{0.7in} p{5.5in}}
Variable &  Categories \\ \hline
SEX &  1 = male, 2 = female \\
RACE& 1 = White alone, 2 = Black or African American alone, 3 = American Indian alone, 4 = other,  5 = two or more races, 6 = Asian alone \\
MAR &  1 = married, 2 = widowed, 3 = divorced, 4 = separated, 5 = never married \\
LANX & 1 = speaks another language, 2 = speaks only English \\
WAOB & born in: 1 = US state, 2 = PR and US island areas, oceania and at sea, 3 = Latin America, 4 = Asia,
5 = Europe, 6 = Africa, 7 = Northern America \\
DIS &  1 = has a disability, 2 = no disability \\
HICOV & 1 = has health insurance coverage, 2 = no coverage \\
MIG & 1 = live in the same house (non movers), 2 = move to outside US and PR, 3 = move to different house in US or PR  \\
SCH & 1 = has not attended school in the last 3 months, 2 = in public school or college, 3 = in private school or college or home school \\
HISP & 1 = not Spanish, Hispanic, or Latino, 2 = Spanish, Hispanic, or Latino  \\ \hline
\end{tabular}
\caption{Variables used in the ACS sample. Data taken
  from the 2012 ACS public use microdata
  samples. In the table, PR stands for Puerto Rico.}
\end{table}

\end{document}
